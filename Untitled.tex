\documentclass[11pt]{article}

    \usepackage[breakable]{tcolorbox}
    \usepackage{parskip} % Stop auto-indenting (to mimic markdown behaviour)
    

    % Basic figure setup, for now with no caption control since it's done
    % automatically by Pandoc (which extracts ![](path) syntax from Markdown).
    \usepackage{graphicx}
    % Maintain compatibility with old templates. Remove in nbconvert 6.0
    \let\Oldincludegraphics\includegraphics
    % Ensure that by default, figures have no caption (until we provide a
    % proper Figure object with a Caption API and a way to capture that
    % in the conversion process - todo).
    \usepackage{caption}
    \DeclareCaptionFormat{nocaption}{}
    \captionsetup{format=nocaption,aboveskip=0pt,belowskip=0pt}

    \usepackage{float}
    \floatplacement{figure}{H} % forces figures to be placed at the correct location
    \usepackage{xcolor} % Allow colors to be defined
    \usepackage{enumerate} % Needed for markdown enumerations to work
    \usepackage{geometry} % Used to adjust the document margins
    \usepackage{amsmath} % Equations
    \usepackage{amssymb} % Equations
    \usepackage{textcomp} % defines textquotesingle
    % Hack from http://tex.stackexchange.com/a/47451/13684:
    \AtBeginDocument{%
        \def\PYZsq{\textquotesingle}% Upright quotes in Pygmentized code
    }
    \usepackage{upquote} % Upright quotes for verbatim code
    \usepackage{eurosym} % defines \euro

    \usepackage{iftex}
    \ifPDFTeX
        \usepackage[T1]{fontenc}
        \IfFileExists{alphabeta.sty}{
              \usepackage{alphabeta}
          }{
              \usepackage[mathletters]{ucs}
              \usepackage[utf8x]{inputenc}
          }
    \else
        \usepackage{fontspec}
        \usepackage{unicode-math}
    \fi

    \usepackage{fancyvrb} % verbatim replacement that allows latex
    \usepackage{grffile} % extends the file name processing of package graphics
                         % to support a larger range
    \makeatletter % fix for old versions of grffile with XeLaTeX
    \@ifpackagelater{grffile}{2019/11/01}
    {
      % Do nothing on new versions
    }
    {
      \def\Gread@@xetex#1{%
        \IfFileExists{"\Gin@base".bb}%
        {\Gread@eps{\Gin@base.bb}}%
        {\Gread@@xetex@aux#1}%
      }
    }
    \makeatother
    \usepackage[Export]{adjustbox} % Used to constrain images to a maximum size
    \adjustboxset{max size={0.9\linewidth}{0.9\paperheight}}

    % The hyperref package gives us a pdf with properly built
    % internal navigation ('pdf bookmarks' for the table of contents,
    % internal cross-reference links, web links for URLs, etc.)
    \usepackage{hyperref}
    % The default LaTeX title has an obnoxious amount of whitespace. By default,
    % titling removes some of it. It also provides customization options.
    \usepackage{titling}
    \usepackage{longtable} % longtable support required by pandoc >1.10
    \usepackage{booktabs}  % table support for pandoc > 1.12.2
    \usepackage{array}     % table support for pandoc >= 2.11.3
    \usepackage{calc}      % table minipage width calculation for pandoc >= 2.11.1
    \usepackage[inline]{enumitem} % IRkernel/repr support (it uses the enumerate* environment)
    \usepackage[normalem]{ulem} % ulem is needed to support strikethroughs (\sout)
                                % normalem makes italics be italics, not underlines
    \usepackage{mathrsfs}
    

    
    % Colors for the hyperref package
    \definecolor{urlcolor}{rgb}{0,.145,.698}
    \definecolor{linkcolor}{rgb}{.71,0.21,0.01}
    \definecolor{citecolor}{rgb}{.12,.54,.11}

    % ANSI colors
    \definecolor{ansi-black}{HTML}{3E424D}
    \definecolor{ansi-black-intense}{HTML}{282C36}
    \definecolor{ansi-red}{HTML}{E75C58}
    \definecolor{ansi-red-intense}{HTML}{B22B31}
    \definecolor{ansi-green}{HTML}{00A250}
    \definecolor{ansi-green-intense}{HTML}{007427}
    \definecolor{ansi-yellow}{HTML}{DDB62B}
    \definecolor{ansi-yellow-intense}{HTML}{B27D12}
    \definecolor{ansi-blue}{HTML}{208FFB}
    \definecolor{ansi-blue-intense}{HTML}{0065CA}
    \definecolor{ansi-magenta}{HTML}{D160C4}
    \definecolor{ansi-magenta-intense}{HTML}{A03196}
    \definecolor{ansi-cyan}{HTML}{60C6C8}
    \definecolor{ansi-cyan-intense}{HTML}{258F8F}
    \definecolor{ansi-white}{HTML}{C5C1B4}
    \definecolor{ansi-white-intense}{HTML}{A1A6B2}
    \definecolor{ansi-default-inverse-fg}{HTML}{FFFFFF}
    \definecolor{ansi-default-inverse-bg}{HTML}{000000}

    % common color for the border for error outputs.
    \definecolor{outerrorbackground}{HTML}{FFDFDF}

    % commands and environments needed by pandoc snippets
    % extracted from the output of `pandoc -s`
    \providecommand{\tightlist}{%
      \setlength{\itemsep}{0pt}\setlength{\parskip}{0pt}}
    \DefineVerbatimEnvironment{Highlighting}{Verbatim}{commandchars=\\\{\}}
    % Add ',fontsize=\small' for more characters per line
    \newenvironment{Shaded}{}{}
    \newcommand{\KeywordTok}[1]{\textcolor[rgb]{0.00,0.44,0.13}{\textbf{{#1}}}}
    \newcommand{\DataTypeTok}[1]{\textcolor[rgb]{0.56,0.13,0.00}{{#1}}}
    \newcommand{\DecValTok}[1]{\textcolor[rgb]{0.25,0.63,0.44}{{#1}}}
    \newcommand{\BaseNTok}[1]{\textcolor[rgb]{0.25,0.63,0.44}{{#1}}}
    \newcommand{\FloatTok}[1]{\textcolor[rgb]{0.25,0.63,0.44}{{#1}}}
    \newcommand{\CharTok}[1]{\textcolor[rgb]{0.25,0.44,0.63}{{#1}}}
    \newcommand{\StringTok}[1]{\textcolor[rgb]{0.25,0.44,0.63}{{#1}}}
    \newcommand{\CommentTok}[1]{\textcolor[rgb]{0.38,0.63,0.69}{\textit{{#1}}}}
    \newcommand{\OtherTok}[1]{\textcolor[rgb]{0.00,0.44,0.13}{{#1}}}
    \newcommand{\AlertTok}[1]{\textcolor[rgb]{1.00,0.00,0.00}{\textbf{{#1}}}}
    \newcommand{\FunctionTok}[1]{\textcolor[rgb]{0.02,0.16,0.49}{{#1}}}
    \newcommand{\RegionMarkerTok}[1]{{#1}}
    \newcommand{\ErrorTok}[1]{\textcolor[rgb]{1.00,0.00,0.00}{\textbf{{#1}}}}
    \newcommand{\NormalTok}[1]{{#1}}

    % Additional commands for more recent versions of Pandoc
    \newcommand{\ConstantTok}[1]{\textcolor[rgb]{0.53,0.00,0.00}{{#1}}}
    \newcommand{\SpecialCharTok}[1]{\textcolor[rgb]{0.25,0.44,0.63}{{#1}}}
    \newcommand{\VerbatimStringTok}[1]{\textcolor[rgb]{0.25,0.44,0.63}{{#1}}}
    \newcommand{\SpecialStringTok}[1]{\textcolor[rgb]{0.73,0.40,0.53}{{#1}}}
    \newcommand{\ImportTok}[1]{{#1}}
    \newcommand{\DocumentationTok}[1]{\textcolor[rgb]{0.73,0.13,0.13}{\textit{{#1}}}}
    \newcommand{\AnnotationTok}[1]{\textcolor[rgb]{0.38,0.63,0.69}{\textbf{\textit{{#1}}}}}
    \newcommand{\CommentVarTok}[1]{\textcolor[rgb]{0.38,0.63,0.69}{\textbf{\textit{{#1}}}}}
    \newcommand{\VariableTok}[1]{\textcolor[rgb]{0.10,0.09,0.49}{{#1}}}
    \newcommand{\ControlFlowTok}[1]{\textcolor[rgb]{0.00,0.44,0.13}{\textbf{{#1}}}}
    \newcommand{\OperatorTok}[1]{\textcolor[rgb]{0.40,0.40,0.40}{{#1}}}
    \newcommand{\BuiltInTok}[1]{{#1}}
    \newcommand{\ExtensionTok}[1]{{#1}}
    \newcommand{\PreprocessorTok}[1]{\textcolor[rgb]{0.74,0.48,0.00}{{#1}}}
    \newcommand{\AttributeTok}[1]{\textcolor[rgb]{0.49,0.56,0.16}{{#1}}}
    \newcommand{\InformationTok}[1]{\textcolor[rgb]{0.38,0.63,0.69}{\textbf{\textit{{#1}}}}}
    \newcommand{\WarningTok}[1]{\textcolor[rgb]{0.38,0.63,0.69}{\textbf{\textit{{#1}}}}}


    % Define a nice break command that doesn't care if a line doesn't already
    % exist.
    \def\br{\hspace*{\fill} \\* }
    % Math Jax compatibility definitions
    \def\gt{>}
    \def\lt{<}
    \let\Oldtex\TeX
    \let\Oldlatex\LaTeX
    \renewcommand{\TeX}{\textrm{\Oldtex}}
    \renewcommand{\LaTeX}{\textrm{\Oldlatex}}
    % Document parameters
    % Document title
    \title{Untitled}
    
    
    
    
    
% Pygments definitions
\makeatletter
\def\PY@reset{\let\PY@it=\relax \let\PY@bf=\relax%
    \let\PY@ul=\relax \let\PY@tc=\relax%
    \let\PY@bc=\relax \let\PY@ff=\relax}
\def\PY@tok#1{\csname PY@tok@#1\endcsname}
\def\PY@toks#1+{\ifx\relax#1\empty\else%
    \PY@tok{#1}\expandafter\PY@toks\fi}
\def\PY@do#1{\PY@bc{\PY@tc{\PY@ul{%
    \PY@it{\PY@bf{\PY@ff{#1}}}}}}}
\def\PY#1#2{\PY@reset\PY@toks#1+\relax+\PY@do{#2}}

\@namedef{PY@tok@w}{\def\PY@tc##1{\textcolor[rgb]{0.73,0.73,0.73}{##1}}}
\@namedef{PY@tok@c}{\let\PY@it=\textit\def\PY@tc##1{\textcolor[rgb]{0.24,0.48,0.48}{##1}}}
\@namedef{PY@tok@cp}{\def\PY@tc##1{\textcolor[rgb]{0.61,0.40,0.00}{##1}}}
\@namedef{PY@tok@k}{\let\PY@bf=\textbf\def\PY@tc##1{\textcolor[rgb]{0.00,0.50,0.00}{##1}}}
\@namedef{PY@tok@kp}{\def\PY@tc##1{\textcolor[rgb]{0.00,0.50,0.00}{##1}}}
\@namedef{PY@tok@kt}{\def\PY@tc##1{\textcolor[rgb]{0.69,0.00,0.25}{##1}}}
\@namedef{PY@tok@o}{\def\PY@tc##1{\textcolor[rgb]{0.40,0.40,0.40}{##1}}}
\@namedef{PY@tok@ow}{\let\PY@bf=\textbf\def\PY@tc##1{\textcolor[rgb]{0.67,0.13,1.00}{##1}}}
\@namedef{PY@tok@nb}{\def\PY@tc##1{\textcolor[rgb]{0.00,0.50,0.00}{##1}}}
\@namedef{PY@tok@nf}{\def\PY@tc##1{\textcolor[rgb]{0.00,0.00,1.00}{##1}}}
\@namedef{PY@tok@nc}{\let\PY@bf=\textbf\def\PY@tc##1{\textcolor[rgb]{0.00,0.00,1.00}{##1}}}
\@namedef{PY@tok@nn}{\let\PY@bf=\textbf\def\PY@tc##1{\textcolor[rgb]{0.00,0.00,1.00}{##1}}}
\@namedef{PY@tok@ne}{\let\PY@bf=\textbf\def\PY@tc##1{\textcolor[rgb]{0.80,0.25,0.22}{##1}}}
\@namedef{PY@tok@nv}{\def\PY@tc##1{\textcolor[rgb]{0.10,0.09,0.49}{##1}}}
\@namedef{PY@tok@no}{\def\PY@tc##1{\textcolor[rgb]{0.53,0.00,0.00}{##1}}}
\@namedef{PY@tok@nl}{\def\PY@tc##1{\textcolor[rgb]{0.46,0.46,0.00}{##1}}}
\@namedef{PY@tok@ni}{\let\PY@bf=\textbf\def\PY@tc##1{\textcolor[rgb]{0.44,0.44,0.44}{##1}}}
\@namedef{PY@tok@na}{\def\PY@tc##1{\textcolor[rgb]{0.41,0.47,0.13}{##1}}}
\@namedef{PY@tok@nt}{\let\PY@bf=\textbf\def\PY@tc##1{\textcolor[rgb]{0.00,0.50,0.00}{##1}}}
\@namedef{PY@tok@nd}{\def\PY@tc##1{\textcolor[rgb]{0.67,0.13,1.00}{##1}}}
\@namedef{PY@tok@s}{\def\PY@tc##1{\textcolor[rgb]{0.73,0.13,0.13}{##1}}}
\@namedef{PY@tok@sd}{\let\PY@it=\textit\def\PY@tc##1{\textcolor[rgb]{0.73,0.13,0.13}{##1}}}
\@namedef{PY@tok@si}{\let\PY@bf=\textbf\def\PY@tc##1{\textcolor[rgb]{0.64,0.35,0.47}{##1}}}
\@namedef{PY@tok@se}{\let\PY@bf=\textbf\def\PY@tc##1{\textcolor[rgb]{0.67,0.36,0.12}{##1}}}
\@namedef{PY@tok@sr}{\def\PY@tc##1{\textcolor[rgb]{0.64,0.35,0.47}{##1}}}
\@namedef{PY@tok@ss}{\def\PY@tc##1{\textcolor[rgb]{0.10,0.09,0.49}{##1}}}
\@namedef{PY@tok@sx}{\def\PY@tc##1{\textcolor[rgb]{0.00,0.50,0.00}{##1}}}
\@namedef{PY@tok@m}{\def\PY@tc##1{\textcolor[rgb]{0.40,0.40,0.40}{##1}}}
\@namedef{PY@tok@gh}{\let\PY@bf=\textbf\def\PY@tc##1{\textcolor[rgb]{0.00,0.00,0.50}{##1}}}
\@namedef{PY@tok@gu}{\let\PY@bf=\textbf\def\PY@tc##1{\textcolor[rgb]{0.50,0.00,0.50}{##1}}}
\@namedef{PY@tok@gd}{\def\PY@tc##1{\textcolor[rgb]{0.63,0.00,0.00}{##1}}}
\@namedef{PY@tok@gi}{\def\PY@tc##1{\textcolor[rgb]{0.00,0.52,0.00}{##1}}}
\@namedef{PY@tok@gr}{\def\PY@tc##1{\textcolor[rgb]{0.89,0.00,0.00}{##1}}}
\@namedef{PY@tok@ge}{\let\PY@it=\textit}
\@namedef{PY@tok@gs}{\let\PY@bf=\textbf}
\@namedef{PY@tok@gp}{\let\PY@bf=\textbf\def\PY@tc##1{\textcolor[rgb]{0.00,0.00,0.50}{##1}}}
\@namedef{PY@tok@go}{\def\PY@tc##1{\textcolor[rgb]{0.44,0.44,0.44}{##1}}}
\@namedef{PY@tok@gt}{\def\PY@tc##1{\textcolor[rgb]{0.00,0.27,0.87}{##1}}}
\@namedef{PY@tok@err}{\def\PY@bc##1{{\setlength{\fboxsep}{\string -\fboxrule}\fcolorbox[rgb]{1.00,0.00,0.00}{1,1,1}{\strut ##1}}}}
\@namedef{PY@tok@kc}{\let\PY@bf=\textbf\def\PY@tc##1{\textcolor[rgb]{0.00,0.50,0.00}{##1}}}
\@namedef{PY@tok@kd}{\let\PY@bf=\textbf\def\PY@tc##1{\textcolor[rgb]{0.00,0.50,0.00}{##1}}}
\@namedef{PY@tok@kn}{\let\PY@bf=\textbf\def\PY@tc##1{\textcolor[rgb]{0.00,0.50,0.00}{##1}}}
\@namedef{PY@tok@kr}{\let\PY@bf=\textbf\def\PY@tc##1{\textcolor[rgb]{0.00,0.50,0.00}{##1}}}
\@namedef{PY@tok@bp}{\def\PY@tc##1{\textcolor[rgb]{0.00,0.50,0.00}{##1}}}
\@namedef{PY@tok@fm}{\def\PY@tc##1{\textcolor[rgb]{0.00,0.00,1.00}{##1}}}
\@namedef{PY@tok@vc}{\def\PY@tc##1{\textcolor[rgb]{0.10,0.09,0.49}{##1}}}
\@namedef{PY@tok@vg}{\def\PY@tc##1{\textcolor[rgb]{0.10,0.09,0.49}{##1}}}
\@namedef{PY@tok@vi}{\def\PY@tc##1{\textcolor[rgb]{0.10,0.09,0.49}{##1}}}
\@namedef{PY@tok@vm}{\def\PY@tc##1{\textcolor[rgb]{0.10,0.09,0.49}{##1}}}
\@namedef{PY@tok@sa}{\def\PY@tc##1{\textcolor[rgb]{0.73,0.13,0.13}{##1}}}
\@namedef{PY@tok@sb}{\def\PY@tc##1{\textcolor[rgb]{0.73,0.13,0.13}{##1}}}
\@namedef{PY@tok@sc}{\def\PY@tc##1{\textcolor[rgb]{0.73,0.13,0.13}{##1}}}
\@namedef{PY@tok@dl}{\def\PY@tc##1{\textcolor[rgb]{0.73,0.13,0.13}{##1}}}
\@namedef{PY@tok@s2}{\def\PY@tc##1{\textcolor[rgb]{0.73,0.13,0.13}{##1}}}
\@namedef{PY@tok@sh}{\def\PY@tc##1{\textcolor[rgb]{0.73,0.13,0.13}{##1}}}
\@namedef{PY@tok@s1}{\def\PY@tc##1{\textcolor[rgb]{0.73,0.13,0.13}{##1}}}
\@namedef{PY@tok@mb}{\def\PY@tc##1{\textcolor[rgb]{0.40,0.40,0.40}{##1}}}
\@namedef{PY@tok@mf}{\def\PY@tc##1{\textcolor[rgb]{0.40,0.40,0.40}{##1}}}
\@namedef{PY@tok@mh}{\def\PY@tc##1{\textcolor[rgb]{0.40,0.40,0.40}{##1}}}
\@namedef{PY@tok@mi}{\def\PY@tc##1{\textcolor[rgb]{0.40,0.40,0.40}{##1}}}
\@namedef{PY@tok@il}{\def\PY@tc##1{\textcolor[rgb]{0.40,0.40,0.40}{##1}}}
\@namedef{PY@tok@mo}{\def\PY@tc##1{\textcolor[rgb]{0.40,0.40,0.40}{##1}}}
\@namedef{PY@tok@ch}{\let\PY@it=\textit\def\PY@tc##1{\textcolor[rgb]{0.24,0.48,0.48}{##1}}}
\@namedef{PY@tok@cm}{\let\PY@it=\textit\def\PY@tc##1{\textcolor[rgb]{0.24,0.48,0.48}{##1}}}
\@namedef{PY@tok@cpf}{\let\PY@it=\textit\def\PY@tc##1{\textcolor[rgb]{0.24,0.48,0.48}{##1}}}
\@namedef{PY@tok@c1}{\let\PY@it=\textit\def\PY@tc##1{\textcolor[rgb]{0.24,0.48,0.48}{##1}}}
\@namedef{PY@tok@cs}{\let\PY@it=\textit\def\PY@tc##1{\textcolor[rgb]{0.24,0.48,0.48}{##1}}}

\def\PYZbs{\char`\\}
\def\PYZus{\char`\_}
\def\PYZob{\char`\{}
\def\PYZcb{\char`\}}
\def\PYZca{\char`\^}
\def\PYZam{\char`\&}
\def\PYZlt{\char`\<}
\def\PYZgt{\char`\>}
\def\PYZsh{\char`\#}
\def\PYZpc{\char`\%}
\def\PYZdl{\char`\$}
\def\PYZhy{\char`\-}
\def\PYZsq{\char`\'}
\def\PYZdq{\char`\"}
\def\PYZti{\char`\~}
% for compatibility with earlier versions
\def\PYZat{@}
\def\PYZlb{[}
\def\PYZrb{]}
\makeatother


    % For linebreaks inside Verbatim environment from package fancyvrb.
    \makeatletter
        \newbox\Wrappedcontinuationbox
        \newbox\Wrappedvisiblespacebox
        \newcommand*\Wrappedvisiblespace {\textcolor{red}{\textvisiblespace}}
        \newcommand*\Wrappedcontinuationsymbol {\textcolor{red}{\llap{\tiny$\m@th\hookrightarrow$}}}
        \newcommand*\Wrappedcontinuationindent {3ex }
        \newcommand*\Wrappedafterbreak {\kern\Wrappedcontinuationindent\copy\Wrappedcontinuationbox}
        % Take advantage of the already applied Pygments mark-up to insert
        % potential linebreaks for TeX processing.
        %        {, <, #, %, $, ' and ": go to next line.
        %        _, }, ^, &, >, - and ~: stay at end of broken line.
        % Use of \textquotesingle for straight quote.
        \newcommand*\Wrappedbreaksatspecials {%
            \def\PYGZus{\discretionary{\char`\_}{\Wrappedafterbreak}{\char`\_}}%
            \def\PYGZob{\discretionary{}{\Wrappedafterbreak\char`\{}{\char`\{}}%
            \def\PYGZcb{\discretionary{\char`\}}{\Wrappedafterbreak}{\char`\}}}%
            \def\PYGZca{\discretionary{\char`\^}{\Wrappedafterbreak}{\char`\^}}%
            \def\PYGZam{\discretionary{\char`\&}{\Wrappedafterbreak}{\char`\&}}%
            \def\PYGZlt{\discretionary{}{\Wrappedafterbreak\char`\<}{\char`\<}}%
            \def\PYGZgt{\discretionary{\char`\>}{\Wrappedafterbreak}{\char`\>}}%
            \def\PYGZsh{\discretionary{}{\Wrappedafterbreak\char`\#}{\char`\#}}%
            \def\PYGZpc{\discretionary{}{\Wrappedafterbreak\char`\%}{\char`\%}}%
            \def\PYGZdl{\discretionary{}{\Wrappedafterbreak\char`\$}{\char`\$}}%
            \def\PYGZhy{\discretionary{\char`\-}{\Wrappedafterbreak}{\char`\-}}%
            \def\PYGZsq{\discretionary{}{\Wrappedafterbreak\textquotesingle}{\textquotesingle}}%
            \def\PYGZdq{\discretionary{}{\Wrappedafterbreak\char`\"}{\char`\"}}%
            \def\PYGZti{\discretionary{\char`\~}{\Wrappedafterbreak}{\char`\~}}%
        }
        % Some characters . , ; ? ! / are not pygmentized.
        % This macro makes them "active" and they will insert potential linebreaks
        \newcommand*\Wrappedbreaksatpunct {%
            \lccode`\~`\.\lowercase{\def~}{\discretionary{\hbox{\char`\.}}{\Wrappedafterbreak}{\hbox{\char`\.}}}%
            \lccode`\~`\,\lowercase{\def~}{\discretionary{\hbox{\char`\,}}{\Wrappedafterbreak}{\hbox{\char`\,}}}%
            \lccode`\~`\;\lowercase{\def~}{\discretionary{\hbox{\char`\;}}{\Wrappedafterbreak}{\hbox{\char`\;}}}%
            \lccode`\~`\:\lowercase{\def~}{\discretionary{\hbox{\char`\:}}{\Wrappedafterbreak}{\hbox{\char`\:}}}%
            \lccode`\~`\?\lowercase{\def~}{\discretionary{\hbox{\char`\?}}{\Wrappedafterbreak}{\hbox{\char`\?}}}%
            \lccode`\~`\!\lowercase{\def~}{\discretionary{\hbox{\char`\!}}{\Wrappedafterbreak}{\hbox{\char`\!}}}%
            \lccode`\~`\/\lowercase{\def~}{\discretionary{\hbox{\char`\/}}{\Wrappedafterbreak}{\hbox{\char`\/}}}%
            \catcode`\.\active
            \catcode`\,\active
            \catcode`\;\active
            \catcode`\:\active
            \catcode`\?\active
            \catcode`\!\active
            \catcode`\/\active
            \lccode`\~`\~
        }
    \makeatother

    \let\OriginalVerbatim=\Verbatim
    \makeatletter
    \renewcommand{\Verbatim}[1][1]{%
        %\parskip\z@skip
        \sbox\Wrappedcontinuationbox {\Wrappedcontinuationsymbol}%
        \sbox\Wrappedvisiblespacebox {\FV@SetupFont\Wrappedvisiblespace}%
        \def\FancyVerbFormatLine ##1{\hsize\linewidth
            \vtop{\raggedright\hyphenpenalty\z@\exhyphenpenalty\z@
                \doublehyphendemerits\z@\finalhyphendemerits\z@
                \strut ##1\strut}%
        }%
        % If the linebreak is at a space, the latter will be displayed as visible
        % space at end of first line, and a continuation symbol starts next line.
        % Stretch/shrink are however usually zero for typewriter font.
        \def\FV@Space {%
            \nobreak\hskip\z@ plus\fontdimen3\font minus\fontdimen4\font
            \discretionary{\copy\Wrappedvisiblespacebox}{\Wrappedafterbreak}
            {\kern\fontdimen2\font}%
        }%

        % Allow breaks at special characters using \PYG... macros.
        \Wrappedbreaksatspecials
        % Breaks at punctuation characters . , ; ? ! and / need catcode=\active
        \OriginalVerbatim[#1,codes*=\Wrappedbreaksatpunct]%
    }
    \makeatother

    % Exact colors from NB
    \definecolor{incolor}{HTML}{303F9F}
    \definecolor{outcolor}{HTML}{D84315}
    \definecolor{cellborder}{HTML}{CFCFCF}
    \definecolor{cellbackground}{HTML}{F7F7F7}

    % prompt
    \makeatletter
    \newcommand{\boxspacing}{\kern\kvtcb@left@rule\kern\kvtcb@boxsep}
    \makeatother
    \newcommand{\prompt}[4]{
        {\ttfamily\llap{{\color{#2}[#3]:\hspace{3pt}#4}}\vspace{-\baselineskip}}
    }
    

    
    % Prevent overflowing lines due to hard-to-break entities
    \sloppy
    % Setup hyperref package
    \hypersetup{
      breaklinks=true,  % so long urls are correctly broken across lines
      colorlinks=true,
      urlcolor=urlcolor,
      linkcolor=linkcolor,
      citecolor=citecolor,
      }
    % Slightly bigger margins than the latex defaults
    
    \geometry{verbose,tmargin=1in,bmargin=1in,lmargin=1in,rmargin=1in}
    
    

\begin{document}
    
    \maketitle
    
    

    
    \hypertarget{ux432ux438ux437ux443ux430ux43bux438ux437ux430ux446ux438ux44f}{%
\section{Визуализация}\label{ux432ux438ux437ux443ux430ux43bux438ux437ux430ux446ux438ux44f}}

    \hypertarget{ux432ux445ux43eux434ux43dux44bux435-ux434ux430ux43dux43dux44bux435}{%
\subsection{Входные
данные}\label{ux432ux445ux43eux434ux43dux44bux435-ux434ux430ux43dux43dux44bux435}}

    \begin{Verbatim}[commandchars=\\\{\}]
--------------------------------------------------------------------------------
-----
| H |  1.0  |  2.0  |  3.0  |  4.0  |  5.0  |  6.0  |  7.0  |  8.0  |  9.0  |
10.0  |
--------------------------------------------------------------------------------
-----
| M |  1.0  |  3.0  |  5.0  |  7.0  |  9.0  | 11.0  | 13.0  | 15.0  | 17.0  |
19.0  |
--------------------------------------------------------------------------------
-----
Температура среды T: 280
    \end{Verbatim}

    \begin{center}
    \adjustimage{max size={0.9\linewidth}{0.9\paperheight}}{Untitled_files/Untitled_3_0.png}
    \end{center}
    { \hspace*{\fill} \\}
    
    \[M_L = \rho\sum\limits_{k=1}^I{\mu_kp_kL(\frac{\mu_0\mu_kH}{k_BT})}\]
Где: * \(\rho\) - концентрация частиц * \(I\) - количество фракций, на
которые разбиваются частицы по размеру * \(p_k=\frac{\rho_k}{\rho}\) -
отношение частиц k-ой фракции к общей концентрации. Тогда
\(\sum\limits_{k=1}^I{p_k} = 1\) * \(L(z)=\coth{z} - \frac{1}{z}\) -
функция Ланжевена * \(\mu_0\) - магнитная проницаемость вакуума *
\(k_B\) - постоянная Больцмана * \(T\) - абсолютная температура

    Приведем величин к безразмерным. Для этого введем реперные значения:

    \begin{Verbatim}[commandchars=\\\{\}]
Реперное значение темературы: 293 K
Реперное значение mu: 1e-19 А²∙м
Реперное значение H: 100 кА/м
Реперное значение ro: 10000000000000000000000 1/м\^{}3
    \end{Verbatim}

    Таблица входных данных с безразмерными величинами:

    \begin{Verbatim}[commandchars=\\\{\}]
--------------------------------------------------------------------------------
-----
| H | 0.01  | 0.02  | 0.03  | 0.04  | 0.05  | 0.06  | 0.07  | 0.08  | 0.09  |
0.1  |
--------------------------------------------------------------------------------
-----
| M |0.05263|0.15789|0.26316|0.36842|0.47368|0.57895|0.68421|0.78947|0.89474|
1.0  |
--------------------------------------------------------------------------------
-----
Реперное значение M: 19.0 кА/м
    \end{Verbatim}

    Приведем формулу к безразмерному виду, где переменные со звездой -
реперные значения, а с волной - абезрамеренные:
\[\tilde{M_L} = \frac{\rho_*\mu_*}{M_*}\tilde{\rho}\sum\limits_{k=1}^I{p_k\tilde{\mu_k}L(\frac{\mu_0\mu_*H_*}{k_BT}\tilde{\mu_k}\tilde{H})}\]
Введем обозначения коэффициентов: * \(a = \frac{\rho_*\mu_*}{M_*}\) *
\(b = \frac{\mu_0\mu_*H_*}{k_BT}\)

Тогда:
\[\tilde{M_L} = a\tilde{\rho}\sum\limits_{k=1}^I{p_k\tilde{\mu_k}L(b\tilde{\mu_k}\tilde{H})}\]
\(\mu_0 = 1.256637 * 10^{-6}\) H/A\^{}2

\(k_B = 1.380649 * 10^{-23}\) Дж/К

    \begin{Verbatim}[commandchars=\\\{\}]
Посчитаем a и b:
a: 52.63157894736842
b: 0.003250637407273163
    \end{Verbatim}

    Разобъем частицы на $I = 6$ фракций.

    
    Будем считать, что $\mu_k$ нам известно. Величины будут располагаться в $[-1.1; 1.3]$ с шагом $0.4$

    
    Получается в исходном абезразмеренном выражении 6 неизвестных:
\(\tilde{\rho}, p_k, k = 1..(I-1)\),
\(p_k = 1 - \sum\limits_{i=1}^{I - 1}{p_i}\)

Проведем минимизацию. Минимизировать будем выражение (\(n\) - количество
входных данных):
\[\sum\limits_{j=1}^{n}{(\tilde{M_L}(H_j) - \tilde{M_j})^2}\]

    \begin{Verbatim}[commandchars=\\\{\}]
m: [-1.1, -0.7000000000000001, -0.30000000000000004, 0.09999999999999998, 0.5,
0.9]
    \end{Verbatim}

    \begin{Verbatim}[commandchars=\\\{\}, frame=single, framerule=2mm, rulecolor=\color{outerrorbackground}]
\textcolor{ansi-red-intense}{\textbf{---------------------------------------------------------------------------}}
\textcolor{ansi-red-intense}{\textbf{TypeError}}                                 Traceback (most recent call last)
Cell \textcolor{ansi-green-intense}{\textbf{In[29], line 78}}
\textcolor{ansi-green}{     75} \def\tcRGB{\textcolor[RGB]}\expandafter\tcRGB\expandafter{\detokenize{0,135,0}}{\textbf{if}} mu\_border \def\tcRGB{\textcolor[RGB]}\expandafter\tcRGB\expandafter{\detokenize{98,98,98}}{==} \def\tcRGB{\textcolor[RGB]}\expandafter\tcRGB\expandafter{\detokenize{98,98,98}}{-}\def\tcRGB{\textcolor[RGB]}\expandafter\tcRGB\expandafter{\detokenize{98,98,98}}{10000}:
\textcolor{ansi-green}{     76}     mu\_border \def\tcRGB{\textcolor[RGB]}\expandafter\tcRGB\expandafter{\detokenize{98,98,98}}{=} mus[\def\tcRGB{\textcolor[RGB]}\expandafter\tcRGB\expandafter{\detokenize{98,98,98}}{1}]
\textcolor{ansi-green-intense}{\textbf{---> 78}} res \def\tcRGB{\textcolor[RGB]}\expandafter\tcRGB\expandafter{\detokenize{98,98,98}}{=} \setlength{\fboxsep}{0pt}\colorbox{ansi-yellow}{dual\_annealing\strut}\setlength{\fboxsep}{0pt}\colorbox{ansi-yellow}{(\strut}\def\tcRGB{\textcolor[RGB]}\expandafter\tcRGB\expandafter{\detokenize{0,135,0}}{\setlength{\fboxsep}{0pt}\colorbox{ansi-yellow}{\textbf{lambda}\strut}}\setlength{\fboxsep}{0pt}\colorbox{ansi-yellow}{ \strut}\setlength{\fboxsep}{0pt}\colorbox{ansi-yellow}{p\strut}\setlength{\fboxsep}{0pt}\colorbox{ansi-yellow}{:\strut}\setlength{\fboxsep}{0pt}\colorbox{ansi-yellow}{ \strut}\setlength{\fboxsep}{0pt}\colorbox{ansi-yellow}{optimization\_func\strut}\setlength{\fboxsep}{0pt}\colorbox{ansi-yellow}{(\strut}\setlength{\fboxsep}{0pt}\colorbox{ansi-yellow}{p\strut}\setlength{\fboxsep}{0pt}\colorbox{ansi-yellow}{,\strut}\setlength{\fboxsep}{0pt}\colorbox{ansi-yellow}{ \strut}\setlength{\fboxsep}{0pt}\colorbox{ansi-yellow}{mus\strut}\setlength{\fboxsep}{0pt}\colorbox{ansi-yellow}{)\strut}\setlength{\fboxsep}{0pt}\colorbox{ansi-yellow}{,\strut}\setlength{\fboxsep}{0pt}\colorbox{ansi-yellow}{ \strut}\setlength{\fboxsep}{0pt}\colorbox{ansi-yellow}{bounds\strut}\setlength{\fboxsep}{0pt}\colorbox{ansi-yellow}{)\strut}
\textcolor{ansi-green}{     79} \def\tcRGB{\textcolor[RGB]}\expandafter\tcRGB\expandafter{\detokenize{0,135,0}}{\textbf{if}} res\def\tcRGB{\textcolor[RGB]}\expandafter\tcRGB\expandafter{\detokenize{98,98,98}}{.}success:
\textcolor{ansi-green}{     80}     optimization\_iterations\def\tcRGB{\textcolor[RGB]}\expandafter\tcRGB\expandafter{\detokenize{98,98,98}}{.}append(OptimizationResult(res\def\tcRGB{\textcolor[RGB]}\expandafter\tcRGB\expandafter{\detokenize{98,98,98}}{.}x, mus))

File \textcolor{ansi-green-intense}{\textbf{c:\textbackslash{}users\textbackslash{}bkmz1\textbackslash{}pycharmprojects\textbackslash{}diplom\textbackslash{}venv\textbackslash{}lib\textbackslash{}site-packages\textbackslash{}scipy\textbackslash{}optimize\textbackslash{}\_dual\_annealing.py:653}}, in \textcolor{ansi-cyan}{dual\_annealing}\textcolor{ansi-blue-intense}{\textbf{(func, bounds, args, maxiter, minimizer\_kwargs, initial\_temp, restart\_temp\_ratio, visit, accept, maxfun, seed, no\_local\_search, callback, x0)}}
\textcolor{ansi-green}{    651} \def\tcRGB{\textcolor[RGB]}\expandafter\tcRGB\expandafter{\detokenize{95,135,135}}{\# Initialization of the energy state}
\textcolor{ansi-green}{    652} energy\_state \def\tcRGB{\textcolor[RGB]}\expandafter\tcRGB\expandafter{\detokenize{98,98,98}}{=} EnergyState(lower, upper, callback)
\textcolor{ansi-green-intense}{\textbf{--> 653}} \setlength{\fboxsep}{0pt}\colorbox{ansi-yellow}{energy\_state\strut}\def\tcRGB{\textcolor[RGB]}\expandafter\tcRGB\expandafter{\detokenize{98,98,98}}{\setlength{\fboxsep}{0pt}\colorbox{ansi-yellow}{.\strut}}\setlength{\fboxsep}{0pt}\colorbox{ansi-yellow}{reset\strut}\setlength{\fboxsep}{0pt}\colorbox{ansi-yellow}{(\strut}\setlength{\fboxsep}{0pt}\colorbox{ansi-yellow}{func\_wrapper\strut}\setlength{\fboxsep}{0pt}\colorbox{ansi-yellow}{,\strut}\setlength{\fboxsep}{0pt}\colorbox{ansi-yellow}{ \strut}\setlength{\fboxsep}{0pt}\colorbox{ansi-yellow}{rand\_state\strut}\setlength{\fboxsep}{0pt}\colorbox{ansi-yellow}{,\strut}\setlength{\fboxsep}{0pt}\colorbox{ansi-yellow}{ \strut}\setlength{\fboxsep}{0pt}\colorbox{ansi-yellow}{x0\strut}\setlength{\fboxsep}{0pt}\colorbox{ansi-yellow}{)\strut}
\textcolor{ansi-green}{    654} \def\tcRGB{\textcolor[RGB]}\expandafter\tcRGB\expandafter{\detokenize{95,135,135}}{\# Minimum value of annealing temperature reached to perform}
\textcolor{ansi-green}{    655} \def\tcRGB{\textcolor[RGB]}\expandafter\tcRGB\expandafter{\detokenize{95,135,135}}{\# re-annealing}
\textcolor{ansi-green}{    656} temperature\_restart \def\tcRGB{\textcolor[RGB]}\expandafter\tcRGB\expandafter{\detokenize{98,98,98}}{=} initial\_temp \def\tcRGB{\textcolor[RGB]}\expandafter\tcRGB\expandafter{\detokenize{98,98,98}}{*} restart\_temp\_ratio

File \textcolor{ansi-green-intense}{\textbf{c:\textbackslash{}users\textbackslash{}bkmz1\textbackslash{}pycharmprojects\textbackslash{}diplom\textbackslash{}venv\textbackslash{}lib\textbackslash{}site-packages\textbackslash{}scipy\textbackslash{}optimize\textbackslash{}\_dual\_annealing.py:175}}, in \textcolor{ansi-cyan}{EnergyState.reset}\textcolor{ansi-blue-intense}{\textbf{(self, func\_wrapper, rand\_gen, x0)}}
\textcolor{ansi-green}{    173} \def\tcRGB{\textcolor[RGB]}\expandafter\tcRGB\expandafter{\detokenize{0,135,0}}{\textbf{if}} \def\tcRGB{\textcolor[RGB]}\expandafter\tcRGB\expandafter{\detokenize{0,135,0}}{self}\def\tcRGB{\textcolor[RGB]}\expandafter\tcRGB\expandafter{\detokenize{98,98,98}}{.}current\_energy \def\tcRGB{\textcolor[RGB]}\expandafter\tcRGB\expandafter{\detokenize{175,0,255}}{\textbf{is}} \def\tcRGB{\textcolor[RGB]}\expandafter\tcRGB\expandafter{\detokenize{0,135,0}}{\textbf{None}}:
\textcolor{ansi-green}{    174}     \def\tcRGB{\textcolor[RGB]}\expandafter\tcRGB\expandafter{\detokenize{0,135,0}}{\textbf{raise}} \def\tcRGB{\textcolor[RGB]}\expandafter\tcRGB\expandafter{\detokenize{215,95,95}}{\textbf{ValueError}}(\def\tcRGB{\textcolor[RGB]}\expandafter\tcRGB\expandafter{\detokenize{175,0,0}}{'}\def\tcRGB{\textcolor[RGB]}\expandafter\tcRGB\expandafter{\detokenize{175,0,0}}{Objective function is returning None}\def\tcRGB{\textcolor[RGB]}\expandafter\tcRGB\expandafter{\detokenize{175,0,0}}{'})
\textcolor{ansi-green-intense}{\textbf{--> 175}} \def\tcRGB{\textcolor[RGB]}\expandafter\tcRGB\expandafter{\detokenize{0,135,0}}{\textbf{if}} (\def\tcRGB{\textcolor[RGB]}\expandafter\tcRGB\expandafter{\detokenize{175,0,255}}{\textbf{not}} \setlength{\fboxsep}{0pt}\colorbox{ansi-yellow}{np\strut}\def\tcRGB{\textcolor[RGB]}\expandafter\tcRGB\expandafter{\detokenize{98,98,98}}{\setlength{\fboxsep}{0pt}\colorbox{ansi-yellow}{.\strut}}\setlength{\fboxsep}{0pt}\colorbox{ansi-yellow}{isfinite\strut}\setlength{\fboxsep}{0pt}\colorbox{ansi-yellow}{(\strut}\def\tcRGB{\textcolor[RGB]}\expandafter\tcRGB\expandafter{\detokenize{0,135,0}}{\setlength{\fboxsep}{0pt}\colorbox{ansi-yellow}{self\strut}}\def\tcRGB{\textcolor[RGB]}\expandafter\tcRGB\expandafter{\detokenize{98,98,98}}{\setlength{\fboxsep}{0pt}\colorbox{ansi-yellow}{.\strut}}\setlength{\fboxsep}{0pt}\colorbox{ansi-yellow}{current\_energy\strut}\setlength{\fboxsep}{0pt}\colorbox{ansi-yellow}{)\strut} \def\tcRGB{\textcolor[RGB]}\expandafter\tcRGB\expandafter{\detokenize{175,0,255}}{\textbf{or}} np\def\tcRGB{\textcolor[RGB]}\expandafter\tcRGB\expandafter{\detokenize{98,98,98}}{.}isnan(
\textcolor{ansi-green}{    176}         \def\tcRGB{\textcolor[RGB]}\expandafter\tcRGB\expandafter{\detokenize{0,135,0}}{self}\def\tcRGB{\textcolor[RGB]}\expandafter\tcRGB\expandafter{\detokenize{98,98,98}}{.}current\_energy)):
\textcolor{ansi-green}{    177}     \def\tcRGB{\textcolor[RGB]}\expandafter\tcRGB\expandafter{\detokenize{0,135,0}}{\textbf{if}} reinit\_counter \def\tcRGB{\textcolor[RGB]}\expandafter\tcRGB\expandafter{\detokenize{98,98,98}}{>}\def\tcRGB{\textcolor[RGB]}\expandafter\tcRGB\expandafter{\detokenize{98,98,98}}{=} EnergyState\def\tcRGB{\textcolor[RGB]}\expandafter\tcRGB\expandafter{\detokenize{98,98,98}}{.}MAX\_REINIT\_COUNT:
\textcolor{ansi-green}{    178}         init\_error \def\tcRGB{\textcolor[RGB]}\expandafter\tcRGB\expandafter{\detokenize{98,98,98}}{=} \def\tcRGB{\textcolor[RGB]}\expandafter\tcRGB\expandafter{\detokenize{0,135,0}}{\textbf{False}}

\textcolor{ansi-red-intense}{\textbf{TypeError}}: ufunc 'isfinite' not supported for the input types, and the inputs could not be safely coerced to any supported types according to the casting rule ''safe''
    \end{Verbatim}


    % Add a bibliography block to the postdoc
    
    
    
\end{document}
